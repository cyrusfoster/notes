\documentclass[10pt,landscape]{article}
\usepackage{multicol}
\usepackage{calc}
\usepackage{ifthen}
\usepackage[landscape]{geometry}
\usepackage{amsmath,amsthm,amsfonts,amssymb}
\usepackage{color,graphicx}  %,overpic}
\usepackage{hyperref}
\usepackage{bm}


\pdfinfo{
  /Title (example.pdf)
  /Creator (TeX)
  /Producer (pdfTeX 1.40.0)
  /Author (Seamus)
  /Subject (Example)
  /Keywords (pdflatex, latex,pdftex,tex)}

% This sets page margins to .5 inch if using letter paper, and to 1cm
% if using A4 paper. (This probably isn't strictly necessary.)
% If using another size paper, use default 1cm margins.
\ifthenelse{\lengthtest { \paperwidth = 11in}}
    { \geometry{top=.5in,left=.5in,right=.5in,bottom=.5in} }
    {\ifthenelse{ \lengthtest{ \paperwidth = 297mm}}
        {\geometry{top=1cm,left=1cm,right=1cm,bottom=1cm} }
        {\geometry{top=1cm,left=1cm,right=1cm,bottom=1cm} }
    }

% Turn off header and footer
\pagestyle{empty}

% Redefine section commands to use less space
\makeatletter
%\renewcommand{\section}{\@startsection{section}{1}{0mm}%
%                                {-1ex plus -.5ex minus -.2ex}%
%                                {0.5ex plus .2ex}%x
%                                {\normalfont\large\bfseries}}
%\renewcommand{\subsection}{\@startsection{subsection}{2}{0mm}%
%                                {-1explus -.5ex minus -.2ex}%
%                                {0.5ex plus .2ex}%
%                                {\normalfont\normalsize\bfseries}}
%\renewcommand{\subsubsection}{\@startsection{subsubsection}{3}{0mm}%
%                                {-1ex plus -.5ex minus -.2ex}%
%                                {1ex plus .2ex}%
%                                {\normalfont\small\bfseries}}
\makeatother

% Define BibTeX command
\def\BibTeX{{\rm B\kern-.05em{\sc i\kern-.025em b}\kern-.08em
    T\kern-.1667em\lower.7ex\hbox{E}\kern-.125emX}}

% Don't print section numbers
\setcounter{secnumdepth}{0}


\setlength{\parindent}{0pt}
\setlength{\parskip}{0pt plus 0.5ex}

%My Environments
\newtheorem{example}[section]{Example}
% -----------------------------------------------------------------------

\renewcommand{\baselinestretch}{1.5}

\begin{document}
\raggedright
\footnotesize
%\begin{multicols}{3}

% multicol parameters
% These lengths are set only within the two main columns
%\setlength{\columnseprule}{0.25pt}
\setlength{\premulticols}{1pt}
\setlength{\postmulticols}{1pt}
\setlength{\multicolsep}{1pt}
\setlength{\columnsep}{2pt}

\subsection{\underline{Kinematics}}

\everymath{\displaystyle}

Basic Kinematic Eq:
$\frac{d}{dt} \bm f = \big[ (\frac{d}{dt})_{rot} + \bm\omega \times \big] \bm f$

Newton's 2nd law:
$\sum \bm F = \frac{d(m \bm v)}{dt}$

$\sum \bm F = m \bm a$ \quad in inertial frame

$\sum \bm M = 
\dot{\bm I} \bm\omega + \bm I \dot{\bm\omega} + \bm\omega \times \bm I \bm\omega$
\quad in body axes, about c.g.

$
\dot{
\begin{bmatrix} \bm r \\ \bm v \\ \bm q \\ \bm \omega \end{bmatrix}}_{13\times1}
=
\begin{bmatrix} \bm v \\ \frac{1}{m} \sum \bm F 
\\ \frac{1}{2} \bm q \begin{bmatrix} 0 \\ \bm\omega \end{bmatrix}_{quat}
\\ \bm I^{-1}(\sum \bm M - \dot{\bm I} \bm\omega - \bm\omega \times \bm I \bm\omega 
\end{bmatrix}
$

\subsection{\underline{Orbits}}

Variation of parameters: 
$\frac{d\alpha}{dt} = \frac{\partial\alpha}{\partial v} a_d$
for orbit element $\alpha$

\subsection{\underline{Math}}

chain rule: for $F(x)=f(g(x))$, $F'(x)=f'(g(x))g'(x)$
or $\frac{\partial y}{\partial x_i} = 
\sum_{l=1}^{m} \frac{\partial y}{\partial u_l}\frac{\partial u_l}{\partial x_i}$

integration by parts: $\int_{a}^{b} u v' = [uv]_{a}^{b} - \int_{a}^{b} u'v$

fundamental lemma of calc of variations:
if $\int_{a}^{b}f(x)h(x) = 0$ for any compactly smooth $h(x)$, then $f(x)=0$


\subsection{\underline{Guidance}}

From optimization problem:

$
\begin{aligned}
& \underset{x, u, t_0 \rightarrow t_1}{\text{minimize}}
& & h(t_1, x) \quad \text{cost function} \\
& \text{subject to}
& & \dot{x} = f(x, u) \quad \text{dynamics constraints} \\
&&& \Phi(t_1, x) = 0 \quad \text{terminal constraints}
\end{aligned}
$

optimality conditions are:
$(1)\ \lambda^T(t) \frac{\partial f}{\partial u} = 0$ \quad
$(2)\ \dot{\lambda}(t) = \lambda^T(t) \frac{\partial f}{\partial x}$ \quad
$(3)\ \frac{\partial h}{\partial x}\Big|_{t_1} + 
\Lambda^T \frac{\partial \Phi}{\partial x}\Big|_{t_1} - 
\lambda(t_1) = 0$

2D flat Earth $\Leftrightarrow$ 2D curved Earth in rotating frame

For uniform gravity:
$x = \begin{bmatrix} r_x \\ r_y \\ v_x \\ v_y \end{bmatrix}$,
$u = \begin{bmatrix} \theta \end{bmatrix}$,
$\dot{x} = f(x, u) = 
\begin{bmatrix} v_x \\ v_y \\ a_T(t)\cos\theta \\ a_T(t)\sin\theta - g
\end{bmatrix}$;
$(1) \& (2) \Rightarrow$
$\tan\theta = \frac{c_2 t + c_4}{c_1 t + c_3}$
for any $\Phi$ or $h$

$
\tan\theta = \left\{
\begin{matrix}
\frac{c_4}{c_3} & & v_x, v_y &
  \text{yaw steering of inclination only} \\
\frac{c_2 t + c_4}{c_3} & \text{if constraining} & r_y, v_x, v_y &
  \text{pitch steering (w/o downrange), yaw steering in insertion plane} \\
\frac{c_2 t + c_4}{c_1 t + c_3} & & r_x, r_y, v_x, v_y &
  \text{yaw steering in local North frame, lunar landing}
\end{matrix}
\right.
$

For inverse square gravity:
$x = \begin{bmatrix} r_x \\ r_y \\ v_x \\ v_y \end{bmatrix}$,
$u = \begin{bmatrix} \theta \end{bmatrix}$,
$\dot{x} = f(x, u) = 
\begin{bmatrix} v_x \\ v_y \\ a_T(t)\cos\theta \\ 
a_T(t)\sin\theta - \frac{GM}{{r_y}^2} \end{bmatrix}$;
$(1) \& (2) \Rightarrow$
$\tan\theta = \frac{1}{\gamma}\frac{c_2 e^{\gamma t}+c_4 e^{-\gamma t}}{c_1 t + c_3}$
with $\gamma = \sqrt{\frac{2GM}{{r_y}^3}}$ 
for any $\Phi$ or $h$

%$
%\tan\theta =
%\begin{cases}
%\frac{c_4}{c_3} & \text{constraining}\ v_x, v_y \\
%\frac{c_2 t + c_4}{c_3} & \text{constrainging}\ r_y, v_x, v_y \\
%\frac{c_2 t + c_4}{c_1 t + c_3} & \text{constraining}\ r_x, r_y, v_x, v_y
%\end{cases}
%$

% You can even have references
%\rule{0.3\linewidth}{0.25pt}
\scriptsize
\bibliographystyle{abstract}
\bibliography{refFile}
%\end{multicols}
\end{document}
