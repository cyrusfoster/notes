\documentclass[11pt,landscape]{article}
\usepackage{multicol}
\usepackage{calc}
\usepackage{ifthen}
\usepackage[landscape]{geometry}
\usepackage{amsmath,amsthm,amsfonts,amssymb}
\usepackage{color,graphicx}  %,overpic}
\usepackage{hyperref}
\usepackage{bm}
\usepackage{gensymb}
\usepackage{tikz}
\usepackage{wrapfig}
\usepackage{graphicx}
\usepackage{import}


\pdfinfo{
  /Title (notes.pdf)
  /Creator (TeX)
  /Producer (pdfTeX 1.40.0)
  /Author ()
  /Subject (Astrodynamics Notes)
  /Keywords (pdflatex, latex,pdftex,tex)}

% This sets page margins to .5 inch if using letter paper, and to 1cm
% if using A4 paper. (This probably isn't strictly necessary.)
% If using another size paper, use default 1cm margins.
\ifthenelse{\lengthtest { \paperwidth = 11in}}
    { \geometry{top=.5in,left=.5in,right=.5in,bottom=.5in} }
    {\ifthenelse{ \lengthtest{ \paperwidth = 297mm}}
        {\geometry{top=1cm,left=1cm,right=1cm,bottom=1cm} }
        {\geometry{top=1cm,left=1cm,right=1cm,bottom=1cm} }
    }


% Turn off header and footer
\pagestyle{empty}

% Redefine section commands to use less space
\makeatletter
%\renewcommand{\section}{\@startsection{section}{1}{0mm}%
%                                {-1ex plus -.5ex minus -.2ex}%
%                                {0.5ex plus .2ex}%x
%                                {\normalfont\large\bfseries}}
%\renewcommand{\subsection}{\@startsection{subsection}{2}{0mm}%
%                                {-1explus -.5ex minus -.2ex}%
%                                {0.5ex plus .2ex}%
%                                {\normalfont\normalsize\bfseries}}
%\renewcommand{\subsubsection}{\@startsection{subsubsection}{3}{0mm}%
%                                {-1ex plus -.5ex minus -.2ex}%
%                                {1ex plus .2ex}%
%                                {\normalfont\small\bfseries}}
\makeatother

% Define BibTeX command
\def\BibTeX{{\rm B\kern-.05em{\sc i\kern-.025em b}\kern-.08em
    T\kern-.1667em\lower.7ex\hbox{E}\kern-.125emX}}

% Don't print section numbers
\setcounter{secnumdepth}{0}


\setlength{\parindent}{0pt}
\setlength{\parskip}{0pt plus 0.5ex}

%My Environments
\newtheorem{example}[section]{Example}
% -----------------------------------------------------------------------

\renewcommand{\baselinestretch}{1.5}

\begin{document}
\linespread{2}
\raggedright
\footnotesize
%\begin{multicols}{3}

% multicol parameters
% These lengths are set only within the two main columns
%\setlength{\columnseprule}{0.25pt}
\setlength{\premulticols}{1pt}
\setlength{\postmulticols}{1pt}
\setlength{\multicolsep}{1pt}
\setlength{\columnsep}{2pt}


%%%%%%%%%%%%%%%%%%%%%%%%%%%%%%%%%%%%%%%%%%%%%%%%%%%%%%%%%%%%%%%%%%%%%%%%%%%%%%%

% Define norm
\newcommand{\norm}[1]{\left\lVert #1 \right\rVert}

% Define \dotp to be dot product symbol
\makeatletter
\newcommand*\dotp{\mathpalette\dotp@{.5}}
\newcommand*\dotp@[2]{\mathbin{\vcenter{\hbox{\scalebox{#2}{$\m@th#1\bullet$}}}}}
\makeatother


%%%%%%%%%%%%%%%%%%%%%%%%%%%%%%%%%%%%%%%%%%%%%%%%%%%%%%%%%%%%%%%%%%%%%%%%%%%%%%%
\newpage
\subsection{\underline{Math}}

\begin{wrapfigure}{r}{0.25\textwidth}
\resizebox{0.2\textwidth}{!}{\import{svg/}{cos_sin.pdf_tex}}

\quad

\import{svg/}{angle_wedge.pdf_tex}

\quad

\import{svg/}{tangential_speed.pdf_tex}
\end{wrapfigure}

Ellipse:
$
e = \sqrt{1-\frac{b^2}{a^2}}
\quad
b = a\sqrt{1-e^2}
\quad
f = 1 - \frac{b}{a}
\quad
e^2 = 2f - f^2
$

Probability: 
OR: add probabilities
AND: multiply probabilities
$P_{collision-total} = 1 - \sum_{i=1}^{n}(1-P_{collision-i})$

MC run with $n$ samples, failure $p$ \& consumer risk $c_r$:
$c_r = \sum_{i=0}^{k} {n \choose i} p^i (1-p)^{n-i}$ (k failures)
$\quad p = 1-c_r^{\frac{1}{n}} \quad n = \frac{\ln c_r}{\ln (1-p)}$ (0 failures)

$
\sigma
= \text{stddev}
= \text{RMS}
= \sqrt{\text{variance}}
$

Pythagore ($C=\frac{\pi}{2}$):
$c^2 = a^2 + b^2
\quad$
Spherical:
$\cos{c} = \cos{a}\cos{b}$

Law of cosines:
$
c^2 = a^2 + b^2 - 2ab\cos{\gamma}
\quad
$
Spherical:
$
\cos{c} = \cos{a}\cos{b} + \sin{a}\sin{b}\cos{C}
$

Law of sines:
$
\frac{a}{\sin{A}} =
\frac{b}{\sin{B}} =
\frac{c}{\sin{C}}
$

$
\cos^{-1} \in [0; \pi]
\quad
\sin^{-1} \in [-\frac{\pi}{2}; \frac{\pi}{2}]
$

$
\mathbf{a} \dotp \mathbf{b} = a b \cos{\theta}
\quad
\lVert \mathbf{a} \times \mathbf{b} \rVert = a b \sin{\theta}
\quad
\theta = \cos^{-1}({\frac{\mathbf{a} \dotp \mathbf{b}}{ab}}) \in [0; \pi]
\quad
$
for sign bit:
$
\text{sign}( \mathbf{N} \dotp (\mathbf{a} \times \mathbf{b})) \in [-\pi; \pi]
$
($\mathbf{N}$ being the reference "North Pole")

Projection of $\bm A$ on $\bm B$:
$proj(\bm A \rightarrow \bm B) = \frac{\bm A \dotp \bm B}{\bm B \dotp \bm B} \bm B$
\quad
$\norm{proj(\bm A \rightarrow \bm B)} = \frac{\bm A \dotp \bm B}{\norm{B}}$

Projection of $\bm A$ on plane defined by normal $\bm N$:
$\bm A - proj(\bm A \rightarrow \bm N)$

Quadratic:
$
ax^2 + bx + c = 0
\quad
\Delta = b^2 - 4ac
\quad
x = \frac{-b \pm \sqrt{\Delta}}{2a}
$

$
\begin{matrix}
\text{Least-squares} & \text{skinny A} & \text{min} \norm{Ax-y} & 
	\begin{matrix} x = (A^TA)^{-1}A^Ty \\ x = (A^TWA)^{-1}A^TWy \end{matrix} &
	W_{ii}=\frac{1}{\sigma_{ii}^2} \quad \text{cov} = P = (A^TWA)^{-1} \\
\text{Least-norm} & \text{fat A} & \text{min} \norm{x}_{y=Ax} & x = A^T(AA^T)^{-1}y \\
	&  & \text{min}\norm{Ax-b}_{Cx=d} & 
	\begin{bmatrix} x \\ \lambda \end{bmatrix} =
	\begin{bmatrix} A^TA & C^T \\ C & 0 \end{bmatrix}^{-1}
	\begin{bmatrix} A^Tb \\ d \end{bmatrix}
\end{matrix}
$

With rotation vector R:
$X^\prime = RX \quad C^\prime = RCR^T$

Tayload expansion:
$
f(x) = f(a) + f'(a)(x-a) + \frac{1}{2}f''(a)(x-a)^2 + HOT
$

Derivatives:
$(fg)^\prime = f^\prime g + fg^\prime$
\quad
$\big( \frac{f}{g} \big)^\prime = \frac{f^\prime g - fg^\prime}{g^2}$

Chain rule: for $F(x)=f(g(x))$, $F'(x)=f'(g(x))g'(x)$
or $\frac{\partial y}{\partial x_i} = 
\sum_{l=1}^{m} \frac{\partial y}{\partial u_l}\frac{\partial u_l}{\partial x_i}$

Integration by parts: $\int_{a}^{b} u v' = [uv]_{a}^{b} - \int_{a}^{b} u'v$

Fundamental lemma of calc of variations:
if $\int_{a}^{b}f(x)h(x) = 0$ for any compactly smooth $h(x)$, then $f(x)=0$




%%%%%%%%%%%%%%%%%%%%%%%%%%%%%%%%%%%%%%%%%%%%%%%%%%%%%%%%%%%%%%%%%%%%%%%%%%%%%%%
\newpage
\subsection{\underline{Orbits}}

\begin{wrapfigure}{r}{0.25\textwidth}
\resizebox{0.25\textwidth}{!}{\import{svg/}{hyperbolic.pdf_tex}}
\end{wrapfigure}

Specific energy:
$
E_y 
= \dfrac{v^2}{2} - \dfrac{\mu}{r} 
= -\dfrac{\mu}{2a} 
= \dfrac{v_{\infty}^2}{2} 
= \dfrac{C3}{2}
\quad
\alpha
= \dfrac{1}{a}
= \dfrac{2}{r} - \dfrac{v^2}{\mu}
= -\dfrac{v_{\infty}^2}{\mu}
= \dfrac{2}{r_p+r_a}
\quad
C3 = -\mu\alpha
$

Semi-major axis:
$
a
= r_p + r_a
= \frac{-\mu}{2 E_y}
$


Along-track maneuver:
\quad
Impulsive:
$\Delta x = 3 \Delta V \Delta t$
\quad
Low-thrust:
$
\Delta x 
= \dfrac{3}{2} \Delta V \Delta t 
= \dfrac{3}{2} a {\Delta t}^2
= \dfrac{3}{2} a \Delta t_{thrust} ( 2\Delta t_{total} - \Delta t_{thurst} )
$

Angular momentum:
$
h
=\norm{R \times V}
=rv \cos \phi
=rv \sin \gamma
=r_p v_p
=r_a v_a
=\sqrt{\mu p}
=r^2 \dot{\nu}
$

Eccentricity:
$
e 
=1-\dfrac{r_p}{a}
=\dfrac{r_a}{a}-1
=\dfrac{r_a-r_p}{r_a+r_p}
=\sqrt{1+\dfrac{2 E_y h^2}{\mu^2}}
=\sqrt{1-\dfrac{\alpha h^2}{\mu}}
=\sqrt{1-\alpha p}
\quad
\vec{e} = \dfrac{1}{\mu} 
\big[ (v^2-\dfrac{\mu}{r})\vec{r} - (\vec{r} \cdot \vec{v})\vec{v} \big]
$

$
r_p = a(1-e)
\quad
r_a = a(1+e)
\quad
p 
= a(1-e^2)
= \dfrac{h^2}{\mu}
= \dfrac{(r_p v_p)^2}{\mu}
$

Period:
$
T=2\pi\sqrt{\dfrac{a^3}{\mu}}
\Leftrightarrow
a = \sqrt[3]{\mu \big( \dfrac{T}{2\pi} \big)^2} 
\quad
\omega 
= \dfrac{2\pi}{T}
= \sqrt{\dfrac{\mu}{a^3}}
= n
\quad
T_{syn}=\dfrac{1}{\big| \dfrac{1}{T_1} - \dfrac{1}{T_2} \big|}
\quad
\Delta t \approx \dfrac{r^2 \Delta\nu}{h}$ for small $\Delta\nu$

Anomaly:
$
r = \dfrac{p}{1+e\cos\nu} = a(1-e\cos E)
\quad
\cos\nu = \dfrac{1}{e} \big( \dfrac{p}{r}-1 \big)
\quad
\cos E = \dfrac{1}{e} \big( 1- \dfrac{r}{a} \big)
= \dfrac{e+\cos\nu}{1+e\cos\nu}
\quad
\dot{r} = \dfrac{\sqrt{\mu p} e \sin\nu}{r(1+e\cos\nu)}
\quad$

Arg of lat: $u = \nu + \omega$
$\quad$
Hyperbolic:
$
\sin \frac{\delta}{2} = \frac{1}{e}
\quad
\Delta = \frac{\mu \sqrt{e^2 - 1}}{v_{\infty}^2}
$

Speed:
$
v_{circ}=\sqrt{\dfrac{\mu}{r}}
\quad
v_{esc}=\sqrt{\dfrac{2\mu}{r}}
\quad
v
= \sqrt{2 \big( \dfrac{\mu}{r} + E_y \big) }
= \sqrt{2\mu \big( \dfrac{1}{r} - \dfrac{1}{2a} \big) }
= \sqrt{\mu \big( \dfrac{2}{r} - \alpha \big) }
$

Plane change:
$
\Delta V = 2 v \sin \big( \dfrac{\theta}{2} \big)
\quad
\theta = 2 sin^{-1} \big( \dfrac{\Delta V}{2v} \big)
$

J2:
$
\dot{\Omega} = - \dfrac{3 n R_E^2 J_2}{2 p^2} \cos i
\Leftrightarrow
\cos i = - \dfrac{2 p^2 \dot{\Omega}}{3 n R_E^2 J_2}
\quad
\dot{\omega} = \dfrac{3 n R_E^2 J_2}{4 p^2} \big( 4 - 5 sin^2 i \big)
$

TOF:
$
\Delta M = n \Delta t
\quad
\Delta t = t - t_0
= \sqrt{\dfrac{a^3}{\mu}} \big[ 2 k \pi + (E-e\sin E) - (E_0-e\sin E_0) \big]
= \sqrt{\dfrac{(-a)^3}{\mu}} \big[ (e\sinh F - F) - (e\sinh F_0 - F_0) \big]
$

3D:
$
\vec{R} = r \begin{bmatrix} \cos\nu & \sin\nu & 0 \end{bmatrix}^T
\quad
\vec{V} = \dfrac{\mu}{p} \begin{bmatrix} -\sin\nu & e+\cos\nu & 0 \end{bmatrix}^T
$

Drag:
$
BC = \dfrac{m}{C_D A}
\quad
acc_{Drag} = \dfrac{\rho v^2}{2 BC}
\quad
\dfrac{da}{dt} = -2 \sqrt{\dfrac{a^3}{\mu}} acc_{Drag}
$

Constants: 
$
r_{Earth} = 6,378.137 $ km$
\quad
r_{GEO} = 42,164 $ km$
\quad
h_{GEO} = 35,786 $ km$
\quad
sma_{Moon} = 384,400 $ km$
\quad
$
Ecliptic $\rightarrow$ Equatorial: $rot_x( \epsilon = 23.44\degree )$


Variation of parameters: 
$\frac{d\alpha}{dt} = \frac{\partial\alpha}{\partial v} a_d$
for orbit element $\alpha$


%%%%%%%%%%%%%%%%%%%%%%%%%%%%%%%%%%%%%%%%%%%%%%%%%%%%%%%%%%%%%%%%%%%%%%%%%%%%%%%
\newpage
\subsection{\underline{Equations of motion}}

\everymath{\displaystyle}

Basic Kinematic Eq:
$\frac{d}{dt} \bm f = \big[ (\frac{d}{dt})_{rot} + \bm\omega \times \big] \bm f$

Newton's 2nd law:
$\sum \bm F = \frac{d(m \bm v)}{dt}$

$\sum \bm F = m \bm a$ \quad in inertial frame

$\sum \bm M = 
\dot{\bm I} \bm\omega + \bm I \dot{\bm\omega} + \bm\omega \times \bm I \bm\omega$
\quad in body axes, about c.g.

$
\dot{
\begin{bmatrix} \bm r \\ \bm v \\ \bm q \\ \bm \omega \end{bmatrix}}_{13\times1}
=
\begin{bmatrix} \bm v \\ \frac{1}{m} \sum \bm F 
\\ \frac{1}{2} \bm q \begin{bmatrix} 0 \\ \bm\omega \end{bmatrix}_{quat}
\\ \bm I^{-1}(\sum \bm M - \dot{\bm I} \bm\omega - \bm\omega \times \bm I \bm\omega 
\end{bmatrix}
$

Rocket equation:
$\Delta V = I_{sp} g_0 \ln\frac{m_0}{m_1}
\quad
m_1 = m_0 e^{-\frac{\Delta V}{I_{sp}g_0}}
\quad
t_{\text{burn}}=\frac{m_0-m_1}{\dot{m}}$

Impulse:
$I = T \Delta t
\quad
\Delta V = \frac{I}{m}
\quad
m_f = \frac{I}{I_{sp}g_0}
$

EP:
$T = \frac{2\eta P}{I_{sp}g_0}$

Thrust:
$T = \dot{m}V_e + (P_{\text{exit}} - P_{\text{ambient}})A_{\text{exit}}
\quad
V_e = I_{sp} g_0
$


%%%%%%%%%%%%%%%%%%%%%%%%%%%%%%%%%%%%%%%%%%%%%%%%%%%%%%%%%%%%%%%%%%%%%%%%%%%%%%%
\newpage
\subsection{\underline{Guidance}}

From optimization problem:

$
\begin{aligned}
& \underset{x, u, t_0 \rightarrow t_1}{\text{minimize}}
& & h(t_1, x) \quad \text{cost function} \\
& \text{subject to}
& & \dot{x} = f(x, u) \quad \text{dynamics constraints} \\
&&& \Phi(t_1, x) = 0 \quad \text{terminal constraints}
\end{aligned}
$

optimality conditions are:
$(1)\ \lambda^T(t) \frac{\partial f}{\partial u} = 0$ \quad
$(2)\ \dot{\lambda}(t) = \lambda^T(t) \frac{\partial f}{\partial x}$ \quad
$(3)\ \frac{\partial h}{\partial x}\Big|_{t_1} + 
\Lambda^T \frac{\partial \Phi}{\partial x}\Big|_{t_1} - 
\lambda(t_1) = 0$

2D flat Earth $\Leftrightarrow$ 2D curved Earth in rotating frame

For uniform gravity:
$x = \begin{bmatrix} r_x \\ r_y \\ v_x \\ v_y \end{bmatrix}$,
$u = \begin{bmatrix} \theta \end{bmatrix}$,
$\dot{x} = f(x, u) = 
\begin{bmatrix} v_x \\ v_y \\ a_T(t)\cos\theta \\ a_T(t)\sin\theta - g
\end{bmatrix}$;
$(1) \& (2) \Rightarrow$
$\tan\theta = \frac{c_2 t + c_4}{c_1 t + c_3}$
for any $\Phi$ or $h$

$
\tan\theta = \left\{
\begin{matrix}
\frac{c_4}{c_3} & & v_x, v_y &
  \text{yaw steering of inclination only} \\
\frac{c_2 t + c_4}{c_3} & \text{if constraining} & r_y, v_x, v_y &
  \text{pitch steering (w/o downrange), yaw steering in insertion plane} \\
\frac{c_2 t + c_4}{c_1 t + c_3} & & r_x, r_y, v_x, v_y &
  \text{yaw steering in local North frame, lunar landing}
\end{matrix}
\right.
$

For inverse square gravity:
$x = \begin{bmatrix} r_x \\ r_y \\ v_x \\ v_y \end{bmatrix}$,
$u = \begin{bmatrix} \theta \end{bmatrix}$,
$\dot{x} = f(x, u) = 
\begin{bmatrix} v_x \\ v_y \\ a_T(t)\cos\theta \\ 
a_T(t)\sin\theta - \frac{GM}{{r_y}^2} \end{bmatrix}$;
$(1) \& (2) \Rightarrow$
$\tan\theta = \frac{1}{\gamma}\frac{c_2 e^{\gamma t}+c_4 e^{-\gamma t}}{c_1 t + c_3}$
with $\gamma = \sqrt{\frac{2GM}{{r_y}^3}}$ 
for any $\Phi$ or $h$



%%%%%%%%%%%%%%%%%%%%%%%%%%%%%%%%%%%%%%%%%%%%%%%%%%%%%%%%%%%%%%%%%%%%%%%%%%%%%%%
\newpage
\subsection{\underline{Time}}

$\text{TAI} = \text{TT} - 32.184s$

$\text{GPS} = \text{TT} - 51.184s$

UT1: smooth, matches Earth's rotation

UTC: Use SI seconds and approximates UT1 with leapseconds


\subsection{\underline{Frames}}

$\text{J2000} \approx \text{ICRF}$

$\text{ECEF} \Leftrightarrow \text{ITRF} \approx \text{ITRS}$

TEME


\subsection{\underline{Matrix}}

\begin{tabular}{ l l l }
Decomposition & Speed & Requirements \\
\hline
LU (gauss-jordan) & $0(\frac{2}{3}n^3)$ & invertible (square) \\ 
QR & $0(\frac{4}{3}n^3)$ & none \\  
Cholesky &  & $\pm$ definite \\
SVD & $0(7n^3)$ & none    
\end{tabular}


%%%%%%%%%%%%%%%%%%%%%%%%%%%%%%%%%%%%%%%%%%%%%%%%%%%%%%%%%%%%%%%%%%%%%%%%%%%%%%%
\newpage
\subsection{\underline{Optimization}}

\text{min} f(x) \text{subject to} c(x)=0

$
\begin{matrix}

 & \text{iterate by solving} & \text{equivalent to finding} \\

n_x = n_c & c_x \Delta x = -c &
\text{roots of } c(x)\\

n_x > n_c & 
\begin{bmatrix} 
f_{xx} + (\lambda^T c)_{xx} & {c_x}^T \\ c_x & 0
\end{bmatrix} 
\begin{bmatrix} 
\Delta x \\ \Delta \lambda
\end{bmatrix} 
= -
\begin{bmatrix} 
f_x + (\lambda^T c)_x \\ c
\end{bmatrix} 
&
\text{saddle point of Lagrangian } L = f(x)+\lambda^T c(x)

\end{matrix}
$



\subsection{\underline{Estimation}}

KF

EKF

UKF

% You can even have references
%\rule{0.3\linewidth}{0.25pt}
\scriptsize
\bibliographystyle{abstract}
\bibliography{refFile}
%\end{multicols}
\end{document}
